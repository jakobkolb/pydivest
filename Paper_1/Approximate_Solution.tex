\section{Approximate Analytical Solution}

Structurally, the model described in Section \ref{sec:Model_Description} consists of a set of coupled ordinary differential equations with algebraic constraints for the economic production process and a stochastic adaptive network process for the social learning component.
We aim to find an analytic description of the dynamics of the model in terms ordinary differential equations of aggregated variables. This can be done in three steps. First, solve the algebraic constraints to the economic production process given by market clearing in the labor market and efficient production in the dirty sector, second use a moment closure to approximate the capital holdings of the heterogeneous households by the moments of their distribution and third use a pair approximation to describe the social learning process in terms of aggregated variables.

\subsection{Algebraic Constraints}

To calculate labor shares $L_c$ and $L_d$ as well as wages in the two sectors, we use equations \eqref{eq:resource_cost} \eqref{eq:equilibrium_wage} and resulting in
\begin{align}
	w &= \frac{\partial Y_d}{\partial L_d} - \frac{\partial c_R}{\partial L_c} \nonumber \\
	&= \frac{\partial Y_d}{\partial L_d} - \frac{\partial c_R}{\partial R} \frac{\partial R}{\partial L_d} \nonumber = \frac{\partial Y_d}{\partial L_d} - \frac{\partial c_R}{\partial R} \frac{\partial}{\partial L_d} \frac{Y_d}{e} \nonumber \\
	&= \frac{\partial Y_d}{\partial L_d} - b_R \frac{\partial}{\partial L_d} \frac{Y_d}{e} = b_d K_d^{\beta_d} \alpha L_d^{\alpha-1}\left( 1-\frac{b_R}{e} \right)
	\label{eq:dirty_wages}
\end{align}
for the dirty sector and
\begin{equation}
	w = b_c K_c^{\beta_c} \alpha L_c^{\alpha-1}
	\label{eq:clean_wages}
\end{equation}
for the clean sector. Combining these results via equation \eqref{eq:population} results in
\begin{equation}
	L = \left( \frac{w}{\alpha} \right)^{\frac{1}{\alpha-1}}\left( \left( b_c K_c^{\beta_c} \right)^{\frac{1}{1-\alpha}} + \left( b_d K_d^{\beta_d} \left( 1 - \frac{b_R}{e} \right)^{\frac{1}{1-\alpha}} \right) \right)
\end{equation}
substituting 
\begin{equation}
	X_c = (b_c K_c^{\beta_c})^{\frac{1}{1-\alpha}}, \qquad X_d = (b_d K_d^{\beta_d})^{\frac{1}{1-\alpha}}, \qquad X_R = \left( 1 - \frac{b_R}{e} \right)^{\frac{1}{1-\alpha}}
	\label{eq:substitutions}
\end{equation}
holds the following result for $w$:
\begin{equation}
	w = \alpha L^{\alpha-1}\left( X_c + X_d X_R \right)^{1-\alpha}.
	\label{eq:wage_result}
\end{equation}
Plugging this into equations \eqref{eq:dirty_wages} and \eqref{eq:clean_wages} results in 
\begin{align}
	L_c &= L \frac{X_c}{X_c + X_d X_R} \label{eq:clean_labor} \\
	L_d &= L \frac{X_d X_R}{X_c + X_d X_R} \label{eq:dirty_labor}
\end{align}
and plugging this into \eqref{eq:resources} results in
\begin{equation}
	R = \frac{b_d}{e}K_d^{\beta_d}L^{\alpha}\left( \frac{X_d X_R}{X_c + X_d X_R} \right)^{\alpha}.
	\label{eq:R_result}
\end{equation}
Using the results for $L_c$ and $L_d$ together with equations \eqref{eq:clean_capital_rent} and \eqref{eq:dirty_capital_rent}<++>, the capital rental rates result in
\begin{align}
	r_c &= \frac{\beta_c}{K_c}X_c L^{\alpha}\left( X_c + X_d X_R \right)^{-\alpha}, \label{eq:r_c_result}\\
	r_d &= \frac{\beta_d}{K_d}X_d X_R L^{\alpha}\left( X_c + X_d X_R \right)^{-\alpha}. \label{eq:r_d_result}
\end{align}
It is also worth noting, that the assumption of zero profits, e.g.
\begin{align}
	Y_c &= w L_c + r_c K_c \nonumber \\
	Y_d &= w L_d + r_d K_d + c_R \nonumber
\end{align}
results in the following restraints for the capital and labor elasticities $\alpha$, $\beta_c$ and $\beta_d$:
\begin{equation}
	\beta_c = \beta_d = 1-\alpha.
	\label{eq:elasticities_restriction}
\end{equation}
To sum up, we solved the algebraic constraints to the ordinary differential equations describing the economic production process resulting in the following equations:
\begin{subequations}
\begin{empheq}[box=\widefbox]{gather}
	X_c = (b_c K_c^{\beta_c})^{\frac{1}{1-\alpha}}, \qquad X_d = (b_d K_d^{\beta_d})^{\frac{1}{1-\alpha}}, \qquad X_R = \left( 1 - \frac{b_R}{e}\frac{G_0^2}{G^2} \right)^{\frac{1}{1-\alpha}}, \\
	w = \alpha L^{\alpha-1}\left( X_c + X_d X_R \right)^{1-\alpha}, \\
	r_c = \frac{\beta_c}{K_c}X_c L^{\alpha}\left( X_c + X_d X_R \right)^{-\alpha}, \\
	r_d = \frac{\beta_d}{K_d}X_d X_R L^{\alpha}\left( X_c + X_d X_R \right)^{-\alpha}, \\
	R = \frac{b_d}{e}K_d^{\beta_d}L^{\alpha}\left( \frac{X_d X_R}{X_c + X_d X_R} \right)^{\alpha}, \\
	\dot{K}_c^{(i)} = s \delta(o_i - c) (r_c K_c^{(i)} + r_d K_d^{(i)} + w L^{(i)}) - \delta K_c^{(i)}, \label{eq:clean_capital_accumulation} \\
	\dot{K}_d^{(i)} = s \delta(o_i - d) (r_c K_c^{(i)} + r_d K_d^{(i)} + w L^{(i)}) - \delta K_d^{(i)}, \label{eq:dirty_capital_accumulation} \\
	\dot{G} = - R, 
\end{empheq}
\end{subequations}

\subsection{Moment Closure}

To describe the capital structure in the model, we use averages that in the limit of $N \rightarrow \infty$ converge the first moments of its distribution as well as the numbers of households investing in clean and dirty capital $N_c$ and $N_d$:
\begin{align}
	\bar{K}_l^{(k)} = \frac{1}{N^{(k)}}\sum_{o_i=k}^{N}K_l^{(i)}, \qquad \lim_{N \rightarrow \infty} \bar{K}_{l}^{(k)} = \braket{K_l^{(i)}}{o_i = k} = \mu_l^{(k)}
	\label{eq:moments_definition}
\end{align}
The time derivative of the means defined in \eqref{eq:moments_definition} is given by capital accumulation \eqref{eq:clean_capital_accumulation} and \eqref{eq:dirty_capital_accumulation} as well as terms resulting from agents switching their savings decisions: 
\begin{equation}
\left.  \begin{aligned}
		\dot{\bar{K}}_c^{(c)} &= (sr_c - \alpha)\bar{K}_c^{(c)} + s r_d \bar{K}_d^{(c)} + w \bar{L} \nonumber \\
		\dot{\bar{K}}_d^{(c)} &= - \alpha\bar{K}_d^{(c)} \nonumber \\
		\dot{\bar{K}}_c^{(d)} &= - \alpha\bar{K}_c^{(d)} \nonumber \\
		\dot{\bar{K}}_d^{(d)} &= sr_c \bar{K}_c^{(d)} + (s r_d - \alpha)\bar{K}_d^{(d)} + w \bar{L} \nonumber
	\end{aligned} \right\} + \rm{switching\ terms}
\end{equation}
Assuming that the average agent changing its savings decision owns the average amount of capital these switching terms are equal to the switching rate $W_{k \rightarrow l}$ times the average capital in the state of origin $K_m^{(k)}$.
\begin{align}
	\dot{\bar{K}}_c^{(c)} &= (sr_c - \alpha)\bar{K}_c^{(c)} + s r_d \bar{K}_d^{(c)} + w \bar{L} + W_{d \rightarrow c} \bar{K}_c^{(d)} - W_{c \rightarrow d} \bar{K}_c^{(c)} \nonumber \\
	\dot{\bar{K}}_d^{(c)} &= - \alpha\bar{K}_d^{(c)} + W_{d \rightarrow c} \bar{K}_d^{(d)} - W_{c \rightarrow d} \bar{K}_d^{(c)} \nonumber \\
	\dot{\bar{K}}_c^{(d)} &= - \alpha\bar{K}_c^{(d)} + W_{c \rightarrow d} \bar{K}_c^{(c)} - W_{d \rightarrow c} \bar{K}_c^{(d)} \nonumber \\
	\dot{\bar{K}}_d^{(d)} &= sr_c \bar{K}_c^{(d)} + (s r_d - \alpha)\bar{K}_d^{(d)} + w \bar{L} + W_{c \rightarrow d} \bar{K}_d^{(c)} - W_{d \rightarrow c} \bar{K}_d^{(d)}
	\label{eq:mean_capital_stocks}
\end{align}

\subsection{Pair Approximation}

To derive a macroscopic approximation of the opinion formation process, we make use of a so called Pair Based Proxy (PBP) that can be derived via pair approximation. This is equivalent to describing the microscopic process in terms of aggregated quantities by making certain assumptions about the properties of their microscopic structure. Therefore, the aggregated quantities of interest are the number of households investing in clean capital $N_c$, the number of households investing in dirty capital $N_d$ as well as the number of links between agents of the same group $[cc]$ and $[dd]$ as well as the number of links in between the two groups $[cd]$. Since the total number of households and links is fixed, these five variables reduce to three degrees of freedom:

\begin{equation}
	X = N^{(c)} - N^{(d)}, \quad Y = [cc] - [dd], \quad Z = [cd]
	\label{eq:opinion_formation_macro_variables}
\end{equation}

These three degrees of freedom span the state space of the investment decisions of the households $\mathbf{S} = (X, Y, Z)^T$. The investment decision making process can be described in terms of jump lengths $\Delta \mathbf{S}_j$ and jump rates $W(\mathbf{S},\mathbf{S} + \Delta \mathbf{S}_j)$ in this state space.
The derivation of these is illustrate by the example of a clean node imitating a dirty node: The rate of this event is given by
\begin{equation}
	W_{c \rightarrow d} = \frac{N}{\tau} (1-\varepsilon) (1 - \varphi) \frac{N^{(c)}}{N}\frac{[cd]}{[cd] + 2 [cc]}\frac{1}{2}\left( \tanh(\bar{C}^{(d)} - \bar{C}^{(d)}) + 1 \right).
	\label{cdswitchingprob}
\end{equation}
In some more detail this results from
\begin{itemize}
	\item $N/\tau$ the rate of social update events i.e. the rate of events per household times the number of household,
	\item $(1-\varepsilon)$ the probability of the event not being a noise event,
	\item $(1-\varphi)$ the probability of imitation events (versus network adaptation events),
	\item $N^{(c)}/N$ the probability of the active node to invest in clean capital,
	\item $[cd]/(2[cc] + [cd])$ the probability of interaction with a node investing in dirty capital,
	\item $\frac{1}{2}\left( \tanh(\bar{C}^{(d)} - \bar{C}^{(c)}) + 1 \right)$ the probability of the active neighbor imitating its neighbor depending on the average difference of income between households investing in clean and dirty capital.
\end{itemize}
The change if state space variables is a little more tricky. Since the event is a clean node imitating a dirty node, we already know about one of the neighbors of the node. Then the remaining neighbors are determined by drawing $k_c - 1$ times from the distribution of neighbors that is assumed to be given by the probability for a neighbor to be dirty $p_d$ or clean $p_c$:
\begin{equation}
	p_c = \frac{2 [cc]}{2[cc] + [cd]}; \qquad p_d = \frac{cc}{2[cc] + [cd]}
	\label{neighbordist}
\end{equation}
Which results in $n_c$ additional clean neighbors and $n_d$ additional dirty neighbors.
\begin{equation}
	n_c = (1-1/k_c)\frac{2[cc]}{N_c}; \quad n_d = (1-1/k_c)\frac{[cd]}{N_c}
	\label{additional_neighbors}
\end{equation}
This results in the following changes in the state space variables:
\begin{align}
	\Delta N_c &= -1 \nonumber \\
	\Delta N_d &= 1 \nonumber \\
	\Delta [cc] &= \left( 1 - \frac{1}{k_c} \right)\frac{2[cc]}{N_c} \nonumber \\
	\Delta [dd] &= \left( 1 - \frac{1}{k_c} \right)\frac{[cd]}{N_c} \nonumber \\
	\Delta [cd] &= -1 + \left( 1 - \frac{1}{k_c} \right)\frac{2[cc] - [cd]}{N_c} \nonumber
\end{align}
So, summing up, the change in the state vector is given by:
\begin{equation}
	\Delta S_{c \rightarrow d} = \colvec{3}{-2}{-k_c}{-1 +  \left( 1 - \frac{1}{k_c} \right)\frac{2[cc] - [cd]}{N_c} }
	\label{cdstatespacechange}
\end{equation}

In terms of these, the dynamics of the PBP can be written as a master equation:

\begin{align}
	\frac{{\rm d} P(\mathbf{S}, t)}{{\rm d} t} = \sum_{j} &P(\mathbf{S} - \Delta \mathbf{S}_j, t) W(\mathbf{S} - \Delta \mathbf{S}_j,\mathbf{S}) \nonumber \\
	&- P(\mathbf{S}, t) W(\mathbf{S},\mathbf{S} + \Delta \mathbf{S}_j) \nonumber
\end{align}


