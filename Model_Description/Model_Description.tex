\section{Abstract}

For the 2 degree (global warming) target target to be viable, most of fossil fuel resources have to stay in the ground. So far, conventional policy options such as cap and trade of emissions or taxations have not proven effective to this point. Therefore, strong hopes lie on social movements as a bottom up process to facilitate norm change with respect to the usage of fossil fuels.
This is difficult to model, since it involves on the one hand macro scale economic dynamics and on the other hand micro scale opinion formation and decision making.

I aim at modeling the impact of a fossil fuel divestment campaign by means of combining a macroeconomic savings model with an opinion formation process amongst heterogeneous households that make bounded rational investment decisions.
This technical paper describes the workings of this model.

\section{Model Description}

The model consists of roughly three kings of processes. One for the economic production process, one for the individual investment decision and one for opinion formation amongst individual about how to make investment decisions.

\subsection{Economic production}

The economic sub-model consists of two sectors. For sake of simplicity, I will call them the \textit{clean} and the \textit{dirty} sector. The dirty sector uses labor $L$, dirty capital $K_d$ and a fossil resource $R$ as input factors, the clean sector uses labor $L$, clean capital $K_c$ and a renewable technology stock $C$ as input factors. The production function in the clean sector is assumed to be
\begin{equation}
	Y_c = b_c L_c^{\alpha_c}K_c^{\beta_c}C^{\gamma_c}, 
	\label{clean_production}
\end{equation}
where $b_c$ is the Solow residual in the clean sector and $\alpha_c$, $\beta_c$ and $\gamma_c$ are the elasticities of the respective input factors.
The production function in the dirty sector is assumed to be
\begin{equation}
	Y_d = {\rm min}\left( b_d L_d^{\alpha_d}K_d^{\beta_d}, e R \right),
	\label{dirty_production}
\end{equation}
where $b_d$ is the Solow residual in the dirty sector, $\alpha_d$ and $\beta_d$ are the input factors of the respective input factors and $e$ is the conversion efficiency of the fossil resource. I chose these production functions because I assume, that input factors in the clean sector can be mutually substituted whereas the fossil resource can not be substituted with capital or labor.
I also assume
\begin{itemize}
	\item a cost for the usage of the fossil resource that depends on the resource use $R$ and the remaining stock of the resource $G$:
		\begin{equation}
			c_R = b_R R^{\rho}\left( \frac{G}{G_0} \right)^{\gamma}
			\label{resource cost}
		\end{equation}
	\item perfect labor mobility and competition for labor between the two sectors, leading to an equilibrium wage that equals the marginal return for labor:
		\begin{equation}
			w = \frac{\partial Y_c}{\partial L_c} = \frac{\partial Y_d}{\partial L_d} - \frac{\partial c_R}{\partial L_d}
			\label{equilibrium_wage}
		\end{equation}
	\item technology specific capital, that can only be used in the sector that it has initially been invested in, leading to independent capital return rates for the two kinds of capital:
		\begin{align}
			r_c &= \frac{\partial Y_c}{\partial K_c} \\
			r_d &= \frac{\partial Y_d}{\partial K_d} - \frac{\partial c_R}{\partial K_d}
		\end{align}
	\item learning by doing with a chance of forgetting in the clean sector, leading to change in the renewable technology stock $C$:
		\begin{equation}
			\dot{C} = Y_c - \chi C
			\label{learning_by_doing}
		\end{equation}
\end{itemize}

\subsection{Investment Decision Making}

The model assumes $N$ heterogeneous households denoted with the index $i$ as owners of labor $L^{(i)}$ and capital $K_c^{(i)}$ and $K_d^{(i)}$ in (potentially) both economic sectors.
From this, households generate income
\begin{equation}
	I^{(i)} = w L^{(i)} + r_c K_c^{(i)} + r_d K_d^{(i)}
	\label{household_income}
\end{equation}
of which they invest a fraction $s$ into one type of capital. I assume that households are bounded rational decision makers using fast and frugal heuristics to make investment decisions.The dominant heuristic for choosing between two options that are comparable on a number of features is called Take The Best. It works as depicted in fig \ref{fig:take_the_best}.

\begin{figure}[h]
	\centering
	\includegraphics[width = 0.4 \textwidth]{figures/TTB.png}
	\caption{Flowchart diagram of the Take The Best heuristic depicting the three building blocks for information search (look for next cue in order), stop of information search (stop as soon as cue discriminates between options) and decision from information at hand (chose option that ranks higher at discriminating cue). For clarification: cues are pieces of information}
	\label{fig:take_the_best}
\end{figure}
The cues that are examined are
\begin{itemize}
	\item the capital return rate, 
	\item the trend of the capital return rate (e.g. the first derivative with respect to time)
	\item the majority of observed behaviors amongst the households acquaintances
	\item a simple label comparison between the two sectors i.e. checking whether one of the options is clean or dirty (this, obviously, always leads to a decision).
\end{itemize}
Since the first two cues are real valued, and would therefore always discriminate, I discriminate only if the higher one exceeds the lower one by more than 10 \%.

For practical reasons, decision heuristics are usually researched in contexts of inferential decisions. Yet they are understood to be applicable for preferential decisions as well. In the case of Take The Best for preferential decision making, I argue that the order in which cues are evaluated can be interpreted as underlying values for of opinions about the decision at hand.

\subsection{Opinion Formation}

Households' opinions (e.g. the order in which they evaluate information to make investment decisions) change as follows: Households form the nodes of a graph of acquaintance relations. Households get active at a constant rate. When a household becomes active, it communicates with one of his acquaintances chosen at random. If they hold the same opinion, nothing happens. If they hold different opinions, one of two things can happen:
\begin{itemize}
	\item Adaptation: with probability $\varphi$, the households end their relation and the active household connects to another household, that holds the same opinion. 
	\item Imitation: with probability $1-\varphi$, an imitation event takes place and the active household assumes the opinion of the other household with probability $p$ increasing with their difference in income.
\end{itemize}
This is closely related to the adaptive voter model, depicted in figure \ref{fig:adaptive_voter_model}

\begin{figure}[h]
	\centering
	\includegraphics[width = 0.9 \textwidth]{figures/AVM.png}
	\caption{The adaptive voter model allows for two types of events upon interaction of an active node with one of its neighbors: \textit{rewiring}, where a node cuts a link with a neighbor of different opinion and creates a new link to a node holding the same opinion and \textit{imitation}, where a node imitates the opinion of a neighbor.}
	\label{fig:adaptive_voter_model}
\end{figure}

I also assume, that a small fraction $\varepsilon$ of events are random, e.g. households doing something else completely, rewiring to a random other household or adopting a random new opinion.

\subsection{Campaign}
How can a fossil fuel divestment campaign be described in the context of this model?
In the early days of the fossil fuel divestment movement, the campaign aimed to convince investors to withdraw funds from companies that used fossil fuel for their operation. Yet it turned out, that this angle did only lead to existing physical capital related to technologies that relied on fossil fuels changed hands to potentially even less environmentally conscious owners.
\par
Therefore, the campaigners changed their strategy to motivating investors to invest only in companies that do not produce more physical capital that relies on fossil fuels (such as construction of power plants, mining for lignite and infrastructure to produce and process crude oil) and to use their existing holdings in fossil fuel companies to lobby for changes to a more sustainable business model.
\par
This, I argue, can be broken down to the following behavior:
\begin{itemize}
	\item campaigners consistently invest in clean capital,
	\item campaigners actively rewire to households holding a different opinion and
	\item campaigners not imitating other opinions, regardless of difference in income.
\end{itemize}
